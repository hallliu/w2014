\documentclass{article}
\usepackage{geometry}
\usepackage[namelimits,sumlimits]{amsmath}
\usepackage{amssymb,amsfonts}
\usepackage{multicol}
\usepackage{graphicx}
\usepackage[cm]{fullpage}
\newcommand{\tab}{\hspace*{5em}}
\newcommand{\conj}{\overline}
\newcommand{\dd}{\partial}
\newcommand{\ep}{\epsilon}
\newcommand{\openm}{\begin{pmatrix}}
\newcommand{\closem}{\end{pmatrix}}
\DeclareMathOperator{\cov}{cov}
\DeclareMathOperator{\rank}{rank}
\DeclareMathOperator{\im}{im}
\DeclareMathOperator{\Span}{span}
\DeclareMathOperator{\Null}{null}
\newcommand{\nc}{\newcommand}
\newcommand{\rn}{\mathbb{R}}
\nc{\cn}{\mathbb{C}}
\nc{\ssn}[1]{\subsubsection*{#1}}
\nc{\inner}[2]{\langle #1,#2\rangle}
\nc{\h}[1]{\widehat{#1}}
\nc{\tl}[1]{\widetilde{#1}}
\nc{\norm}[1]{\left\|{#1}\right\|}
\nc{\lb}{\lambda}
\nc{\ddx}[1]{\frac{d}{d{#1}}}
\nc{\ddxt}[1]{\frac{d^2}{d{#1}^2}}
\begin{document}

Name: Hall Liu

Date: \today 
\vspace{1.5cm}

\subsection*{1}
\ssn{a}
The first derivative of $\phi_2$ with respect to $\lb$ is $\frac{\ddx{\lb}\|p(\lb)\|}{\|p(\lb)\|^2}$. Differentiating again gives a $\|p(\lb)\|^2$ in the denominator (which we don't care about because it doesn't change sign), and the numerator is
\[ \|p(\lb)\|^2\ddxt{\lb}\|p(\lb)\|-\left(\ddx{\lb}\|p(\lb)\|\right)^2\cdot2\|p(\lb)\|=\|p(\lb)\|\left(\|p(\lb)\|\ddxt{\lb}\|p(\lb)\|-2\left(\ddx{\lb}\|p(\lb)\|\right)^2\right)\]
From (4.39) in the text, we have $\|p(\lb)\|=\sqrt{\sum\frac{(q_i^Tg)^2}{(\lb_i+\lb)^2}}$. Since $(q_i^Tg)^2$ is constant wrt $\lb$, denote it by $b_i$. Then, we have
\[\ddx{\lb}\|p(\lb)\|=-\frac{\sum\frac{b_i}{(\lb_i+\lb)^3}}{\|p(\lb)\|}\]
Taking the second derivative, we have
\[-\frac{1}{\|p(\lb)\|}\cdot-3\sum\frac{b_i}{(\lb_i+\lb)^4}+\frac{1}{\|p(\lb)\|^2}\left(\sum\frac{b_i}{(\lb_i+\lb)^3}\right)\frac{\sum\frac{b_i}{(\lb_i+\lb)^3}}{\|p(\lb)\|}\]
Simplifying, this becomes
\[\frac{3}{\|p(\lb)\|}\sum\frac{b_i}{(\lb_i+\lb)^4}+\frac{1}{\|p(\lb)\|}\left(\ddx{\lb}\|p(\lb)\|\right)^2\]
so the expression inside the parens in the numerator of the second derivative of $\phi_2$ is
\[3\sum\frac{b_i}{(\lb_i+\lb)^4}-\frac{1}{\|p(\lb)\|^2}\left(\sum\frac{b_i}{(\lb_i+\lb)^3}\right)^2\]
Thus, the second derivative will be nonnegative everywhere if we have that 
\[\left(\sum\frac{b_i}{(\lb_i+\lb)^3}\right)^2-\left(\sum\frac{b_i}{(\lb_i+\lb)^4}\right)\left(\sum\frac{b_i}{(\lb_i+\lb)^2}\right)\leq0\]
Expanding out the products and subtracting, the remaining terms are of the form 
\[\frac{2b_ib_j}{(\lb_i+\lb)^3(\lb_j+\lb)^3}-\frac{b_ib_j}{(\lb_i+\lb)^4(\lb_j+\lb)^2}-\frac{b_ib_j}{(\lb_i+\lb)^4(\lb_j+\lb)^2}\]
Factoring out the $b_ib_j$ (they're all positive), we're left with an expression of the form $2a^3b^3-a^4b^2-a^2b^4$, where $a,b\geq0$. In turn, this can be written as $-(a-b)^2a^2b^2\leq0$, which means that the whole thing is less than $0$, which means that the second derivative is nonnegative.
\ssn{b}
%Expressing the ``tangent below the graph'' property in the terms of this problem, we have that $\phi_2(\lb_l)-\phi_2'(\lb_l)(\lb_{l+1}-\lb_l)\leq\phi_2(\lb_{l+1})$. Substituting in $\lb_{l+1}=\lb_l-\frac{\phi_2(\lb_l)}{\phi_2'(\lb_l)}$, we have
%\[\phi_2(\lb_l)-\phi_2'(\lb_l)\left(-\frac{\phi_2(\lb_l)}{\phi_2'(\lb_l)}\right)\leq\phi_2(\lb_{l+1})\]
We know that $\phi_2$ is nonincreasing, so $\phi_2'(\lb)\leq0$ everywhere. If $\lb_{l}<\lb^*$, then $\phi_2(\lb_l)>0$, so $\lb_{l+1}=\lb_l-\frac{\phi_2(\lb_l)}{\phi_2'(\lb_l)}>\lb_l$. Then, using the ``tangent-below-the-graph'' property $\phi_2(\lb_l)+\phi_2'(\lb_l)(\lb^*-\lb_l)\leq\phi_2(\lb^*)=0$, we have that $\phi_2'(\lb_l)\leq-\frac{\phi_2(\lb_l)}{\lb^*-\lb_l}$, so that $\lb_l-\frac{\phi_2(\lb_l)}{\phi_2'(\lb_l)}\leq\lb_l+\frac{\phi_2(\lb_l)}{\frac{\phi_2(\lb_l)}{\lb^*-\lb_l}}=\lb^*$.

Otherwise if $\lb_l>\lb^*$, then the inequality $\phi_2(\lb_l)+\phi_2'(\lb_l)(\lb^*-\lb_l)\leq0$ turns into $\phi_2'(\lb_l)\geq-\frac{\phi_2(\lb_l)}{\lb^*-\lb_l}$. Since we now have $\phi_2(\lb_l)<0$, then $\lb_l-\frac{\phi_2(\lb_l)}{\phi_2'(\lb_l)}\leq\lb_l+\frac{\phi_2(\lb_l)}{\frac{\phi_2(\lb_l)}{\lb^*-\lb_l}}=\lb^*$
\ssn{c}
First, note that the ``otherwise'' condition will be hit only finitely times -- if the initial point is to the right of $\lb^*$, then that condition will be executed until $\tl{\lb}^{l+1}$ ends up to the right of $-\lb_1$ or until the iteration of the second condition takes $\lb^l$ to the left of $\lb^*$ (which is guaranteed to occur in a finite number of steps). Then, once the point is in $(-\lb_1,lb^*)$, all subsequent points will lie in that interval also by (b). Thus, all we need to worry about in the tail of the sequence is the Newton's method updating.

We already know that the algorithm converges to something in $(-\lb_1, \lb^*]$ due to the sequence being monotonic and bounded. Suppose that the sequence converges to something less than $\lb^*$, call it $x^*$. Let $\phi_2(x^*)=M>0$, and let $N$ be a uniform bound on $|\phi_2'(\lb)|$ on some open interval $I$ containing $x^*$. Then, we have for any $\lb^l\in I$, $\lb^l<x^*$ that $\lb^{l+1}=\lb^l+\frac{\phi_2(\lb^l)}{|\phi_2(\lb^l)|}\geq\lb^l+\frac{M}{N}$. Then, for $\lb^l$ sufficiently close to $x^*$, $\lb^{l+1}>x^*$, contradicting convergence.

To show quadratic convergence, for any $l$, we have by Taylor's theorem that
\[0=\phi_2(\lb^*)=\phi_2(\lb^l+(\lb^*-\lb^l))=\phi_2(\lb^l)+\phi_2'(\lb^l)(\lb^*-\lb^l)+\frac{1}{2}(\lb^*-\lb^l)^2\phi_2''(t)\]
for some $t\in[\lb^l,\lb^*]$. Then, dividing by $\phi_2'(\lb^l)(\lb^*-\lb^l)^2$ and rearranging gives
\[0=\frac{\frac{\phi_2(\lb^l)}{\phi_2'(\lb^l)}+\lb^*-\lb^l}{(\lb^*-\lb^l)^2}+\frac{\phi_2''(t)}{\phi_2'(\lb^l)}\]
The first term is the ratio that we care about -- we want to show that it's bounded above. Fortunately, the second derivative is uniformly bounded above on $(\lb^k, lb^*)$ where $\lb^k$ is the first iterate in $(-\lb_1,lb^*)$, and the magnitude of first derivative is uniformly bounded below on the same interval. Thus, we have quadratic convergence.
\ssn{d}



\end{document}

