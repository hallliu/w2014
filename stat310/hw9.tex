\documentclass{article}
\usepackage{geometry}
\usepackage[namelimits,sumlimits]{amsmath}
\usepackage{amssymb,amsfonts}
\usepackage{multicol}
\usepackage{graphicx}
\usepackage{mathrsfs}
\usepackage[cm]{fullpage}
\newcommand{\tab}{\hspace*{5em}}
\newcommand{\conj}{\overline}
\newcommand{\dd}{\partial}
\newcommand{\ep}{\epsilon}
\newcommand{\openm}{\begin{pmatrix}}
\newcommand{\closem}{\end{pmatrix}}
\DeclareMathOperator{\cov}{cov}
\DeclareMathOperator{\rank}{rank}
\DeclareMathOperator{\im}{im}
\DeclareMathOperator{\Span}{span}
\DeclareMathOperator{\Null}{null}
\newcommand{\nc}{\newcommand}
\newcommand{\rn}{\mathbb{R}}
\nc{\cn}{\mathbb{C}}
\nc{\ssn}[1]{\subsubsection*{#1}}
\nc{\inner}[2]{\langle #1,#2\rangle}
\nc{\h}[1]{\widehat{#1}}
\nc{\tl}[1]{\widetilde{#1}}
\nc{\norm}[1]{\left\|{#1}\right\|}
\begin{document}

Name: Hall Liu

Date: \today 
\vspace{1.5cm}
\subsection*{2}
Let $d$ be an arbitrary point in $\mathscr{F}(x^*)$. We want to show that $d$ is also in the tangent cone.

First, if we replace $\mathscr{A}(x^*)$ in the argument in the book with the matrix of equality constraint gradients $\mathscr{E}(x^*)$, and follow the argument until (12.42b), we obtain a feasible sequence of $z_k$ defined via the implicit function theorem by $z(t)$ as a solution to (12.40). This is possible because the MFCQ condition specifies that the equality constraint gradients are linearly independent. 

In addition, since the function and its constraints are smooth, the implicit function theorem also gives us an expression for $z'(0)$, which is the inverse Jacobian of $R$ at $t=0$ multiplied by the derivative of $R$ by $t$ at $t=0$. In other words, 
\[z'(0)=-\openm \mathscr{E}(x^*)\\Z^T\closem^{-1}\openm-\mathscr{E}(x^*)d\\-Z^Td\closem\]
Clearly, $d$ is a solution to this system. Since the Jacobian is invertible, it is the unique solution, so we have that $z'(0)=d$.

Now, let $w$ be some vector from the MFCQ conditions where $\mathscr{A}(x^*)w$ is positive in all its entries. For any $\ep>0$, we have that $d+\ep w\in\mathscr{F}(x^*)$, since $\ep w\in\mathscr{F}(x^*)$ by the form of the MFCQ conditions and the sum of two vectors in a cone is still in the cone. Then, there exists a curve $z_\ep(t)$ which is feasible for small $t$, with 
\[z_\ep'(0)=d+\ep w=\lim_{t\to 0}\frac{z_\ep(t)-z_\ep(0)}{t}\]
Note that $z_\ep(0)=x^*$ by the construction in the book, so this means that $d+\ep w$ is actually in the tangent cone for all $\ep>0$, which means that $d$ is a limit point of the tangent cone. Since the tangent cone is closed, $d$ is in the tangent cone.

\end{document}
