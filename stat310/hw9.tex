\documentclass{article}
\usepackage{geometry}
\usepackage[namelimits,sumlimits]{amsmath}
\usepackage{amssymb,amsfonts}
\usepackage{multicol}
\usepackage{graphicx}
\usepackage{mathrsfs}
\usepackage[cm]{fullpage}
\newcommand{\tab}{\hspace*{5em}}
\newcommand{\conj}{\overline}
\newcommand{\dd}{\partial}
\newcommand{\ep}{\epsilon}
\newcommand{\openm}{\begin{pmatrix}}
\newcommand{\closem}{\end{pmatrix}}
\DeclareMathOperator{\cov}{cov}
\DeclareMathOperator{\rank}{rank}
\DeclareMathOperator{\im}{im}
\DeclareMathOperator{\Span}{span}
\DeclareMathOperator{\Null}{null}
\newcommand{\nc}{\newcommand}
\newcommand{\rn}{\mathbb{R}}
\nc{\cn}{\mathbb{C}}
\nc{\ssn}[1]{\subsubsection*{#1}}
\nc{\inner}[2]{\langle #1,#2\rangle}
\nc{\h}[1]{\widehat{#1}}
\nc{\tl}[1]{\widetilde{#1}}
\nc{\norm}[1]{\left\|{#1}\right\|}
\begin{document}

Name: Hall Liu

Date: \today 
\vspace{1.5cm}
\subsection*{2}
Let $d$ be an arbitrary point in $\mathscr{F}(x^*)$. We want to show that $d$ is also in the tangent cone.

First, if we replace $\mathscr{A}(x^*)$ in the argument in the book with the matrix of equality constraint gradients $\mathscr{E}(x^*)$, and follow the argument until (12.42b), we obtain a feasible sequence of $z_k$ defined via the implicit function theorem by $z(t)$ as a solution to (12.40). This is possible because the MFCQ condition specifies that the equality constraint gradients are linearly independent. 

In addition, since the function and its constraints are smooth, the implicit function theorem also gives us an expression for $z'(0)$, which is the inverse Jacobian of $R$ at $t=0$ multiplied by the derivative of $R$ by $t$ at $t=0$. In other words, 
\[z'(0)=-\openm \mathscr{E}(x^*)\\Z^T\closem^{-1}\openm-\mathscr{E}(x^*)d\\-Z^Td\closem\]
Clearly, $d$ is a solution to this system. Since the Jacobian is invertible, it is the unique solution, so we have that $z'(0)=d$.

Now, let $w$ be some vector from the MFCQ conditions where $\mathscr{A}(x^*)w$ is positive in all its entries. For any $\ep>0$, we have that $d+\ep w\in\mathscr{F}(x^*)$, since $\ep w\in\mathscr{F}(x^*)$ by the form of the MFCQ conditions and the sum of two vectors in a cone is still in the cone. Then, there exists a curve $z_\ep(t)$ which is feasible for small $t$, with 
\[z_\ep'(0)=d+\ep w=\lim_{t\to 0}\frac{z_\ep(t)-z_\ep(0)}{t}\]
Note that $z_\ep(0)=x^*$ by the construction in the book, so this means that $d+\ep w$ is actually in the tangent cone for all $\ep>0$, which means that $d$ is a limit point of the tangent cone. Since the tangent cone is closed, $d$ is in the tangent cone.

For the particular problem, the $c_j$ are functions $z-\cos(2\pi j/n)x-\sin(2\pi j/n)y$ for $n\geq5$. The gradient of this is the vector $\openm -\cos(2\pi j/n)&-\sin(2\pi j/n)&1\closem$. For any $x$, if we take $w$ to be the vector with a $1$ in its third entry and zero elsewhere, then we satisfy the requirement that $\nabla c_j(x)^Tw=1>0$. Since there are no equality constraints, this means that the MFCQ conditions are satisfied, and the tangent cone and the feasible set are equal for any $x$ for this function.
\subsection*{3}
\ssn{a}
The gradient of the first constraint is $\openm-3(1-x_1)^2&-1\closem$ (evaluating to $\openm -3&-1\closem$ at $x^*$), and the gradient of the second constraint is $\openm0.5x_1&1\closem$ (evaluating to $\openm 0&1\closem$ at $x^*$). These two are linearly independent, so LICQ is satisfied.
\ssn{b}
The Lagrangian of this problem is $-2x_1+x_2-\lambda_1((1-x_1)^3-x_2)-\lambda_2(x_2+0.25x_1^2-1)$. Its gradient is
\[\openm-2+3\lambda_1(1-x_1)^2-0.5\lambda_2x_1\\1+\lambda_1-\lambda_2\closem\]
Plugging in $x^*$ gives $\openm 3\lambda_1-2\\1+\lambda_1-\lambda_2\closem$, so if we take $\lambda^*=\openm2/3\\5/3\closem$, the Lagrangian will be zero. Since we know that all the constraints are zero at $x^*$ and both $\lambda$s are positive, the KKT conditions are satisfied.
\ssn{c}
The set $\mathscr{F}(x^*)$ is the set of all points $d$ for which $d^T\openm-3\\-1\closem\geq0$ and $d^T\openm0\\1\closem\geq0$. This is a narrow cone delimited by the negative $y$-axis and the ray defined by $(1,-3)$. 

The set $\mathscr{C}(x^*, \lambda^*)$ is the union of the two half-lines generated by $(0, -1)$ and $(1, -3)$, since both components of $\lambda^*$ are positive.
\ssn{d}
The Hessian of the Lagrangian is the matrix
\[\openm -6\lambda_1(1-x_1)&-0.5x_2\\0&0\closem=\openm-4&-0.5\\0&0\closem\]
at $x^*, \lambda^*$. 
If we take $w=(0\ -1)^T$, then the value in question ($w^THw$) is zero. If we take $w=(1\ -3)^T$, then the value in question is also zero. Thus, the necessary conditions are satisfied, but the sufficient ones are not.
\subsection*{4}
Let $B$ be the block matrix $\openm A&I\closem$, where $I$ is the $n\times n$ identity. Then, let $K$ be the cone $\{Bx|x\geq0\}$. By Farkas', either $b\in K$ xor there exists some $y$ with $b^Ty<0$ and $B^Ty\geq0$. First, if $b\in K$, then there exists some $x=\openm x_1\\x_2\closem\geq0$ with $b=Bx=Ax_1+x_2$. Since $x_2\geq0$, this means that $b\geq Ax_1$ with $x_1\geq0$, so the first condition we have is implied. Conversely, if we suppose that there is some $x_1$ with $b\geq Ax_1$ and $x_1\geq0$, then let $x=\openm x_1\\b-Ax_1\closem\geq0$. Then $Bx=Ax_1+b-Ax_1=b$, so $b\in K$. 

For the second possibility, $B^Ty\geq0$ is equivalent to $\openm A^Ty\\y\closem\geq0$, which is in turn equivalent to $A^Ty\geq0$ and $y\geq0$. Along with $b^Ty<0$, this is equivalent to the second condition. Thus, since the two conditions we have correspond exactly to the mutual exclusion specified by Farkas', then these two conditions are mutually exclusive and one must always hold.
\end{document}
